\documentclass[10pt]{article}
\usepackage[paperheight=25.5 cm ,paperwidth=14.25cm,left=1.5cm,right=1.5cm,top=1.5cm,bottom=1.5cm]{geometry}
\usepackage{amsmath, amssymb, amsfonts, esdiff, esint, indentfirst, textcomp, longtable }
\usepackage[]{graphicx}
\usepackage{subcaption}
\usepackage[]{subfiles}
\usepackage[]{caption}

\usepackage[]{svg}
\usepackage{ifluatex}
\ifluatex
  \usepackage{pdftexcmds}
  \makeatletter
  \let\pdfstrcmp\pdf@strcmp
  \let\pdffilemoddate\pdf@filemoddate
  \makeatother
\fi 

\begin{document}
\subsection*{Appendix: Tugas 1 Gelombang 29 Agustus 2024}
\subsubsection*{Soal 1: Massa dalam pendulum.} Dari gambar 1, berapkah frekuensi sudut sistem dan persamaan yang harus dipenuhi sehingga sistem memiliki periode sebesar 1 detik?
\subsubsection*{Soal 2: Massa pada tali.} Dari gambar 1, berapakah frekuensi sudut sistem?
\begin{figure}[h]
    \centering
    \includesvg[height=0.4\textwidth, inkscape=force]{../Rss/Waves/Appen/GHSPend.svg}
    \includesvg[width=0.4\textwidth, inkscape=force]{../Rss/Waves/Appen/GHSTali.svg}
    \caption*{Gambar 1: GHS pada ayunan massa (soal 1) dan massa pada tali (soal 2).}
  \end{figure}

\end{document}