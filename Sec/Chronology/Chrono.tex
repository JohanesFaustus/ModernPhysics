\documentclass[../main.tex]{subfiles}
\begin{document}
\begin{center}
    \begin{longtable}{p{0.05\textwidth}  p{0.85\textwidth}}
    1632&Galileo's Dialogo sopra i due massimi sistemi del mondo (Dialogue Concerning the Two Chief World Systems) contains the principle that we today call Galilean (or classical) relativity. This principle will become one of the cornerstones of Einstein's relativity.\\
1687&Isaac Newton's Philosophiæ Naturalis Principia Mathematica (The Mathematical Principles of Natural  Philosophy) lays out his laws of motion.(Fun fact: Newton wrote $F = p'$, not $F = ma$.)\\
1804&Thomas Young's paper “Experiments and Calculations Relative to Physical Optics” describes an early form of the double-slit experiment that demonstrates the wave nature of light.\\
1861&James Maxwell's paper “On Physical Lines of Force” lays out a set of equations summarizing classical electromagnetism. Maxwell's roughly 20 equations were later combined into 4 by Oliver Heaviside.\\
1887&Albert Michelson and Edward Morley's paper “On the Relative Motion of the Earth and the Luminiferous Ether” describes their failed attempt to detect the Earth's motion through the aether.\\
1889&Johannes Rydberg's paper “Researches sur la constitution des spectres d'émission des éléments chimiques” (“Research on the Constitution of the Emission Spectra of Chemical Elements”) gives a formula for the spectral lines of hydrogen and hydrogen-like atoms.\\
1897&In two papers, both entitled “Cathode Rays,” J.J. Thomson announces the discovery—and measures the mass and charge—of very light, negatively charged particles that are parts of atoms. (Thomson used the word “corpuscles” for what we now call electrons.) \\
1900&Max Planck proposes quantization of cavity radiation to resolve the ultraviolet catastrophe. (He developed the ideas behind this proposal in a series of papers and talks in 1900 and 1901.)\\
1905&Albert Einstein's paper “Zur Elektrodynamik bewegter Körper” (“On the Electrodynamics of Moving Bodies”) introduces his special theory of relativity. Later that year he publishes “Ist die Trägheit eines Körpers von seinem Energieinhalt abhängig?” (“Does the Inertia of a Body Depend Upon Its Energy Content?”) which introduces an early form of the relation $E = mc^2$.\\
1905&Einstein's paper “Über einem die Erzeugung und Verwandlung des Lichtes betreffenden heuristischen Gesichtspunkt” (“On a Heuristic Viewpoint Concerning the Production and Transformation of Light”) shows that the quantization of light, earlier proposed by Planck, explains the photoelectric effect.\\
1909&G.I. Taylor's paper “Interference Fringes with Feeble Light” describes performing the double-slit experiment with light of such low intensity that non-overlapping incident regions appear on the back wall, eventually forming an interference pattern.\\
1911&Ernest Rutherford's paper “The Scattering of $\alpha$ and $\beta$ Particles by Matter and the Structure of the Atom” proposes a nucleus of positive charge at the core of an atom, based on the gold foil experiments performed by Geiger and Marsden.\\
1912&Henrietta Swan Leavitt publishes “Periods of 25 Variable Stars in the Small Magellanic Cloud,” giving a relationship between the oscillation period of Cepheid variable stars and their intrinsic brightness. This relationship was later used by Edwin Hubble to show that other galaxies exist, and later to discover the expansion of the universe. (Leavitt's paper was signed by her supervisor Edward Pickering, with a note that Leavitt had “prepared” the paper.)\\
1913&In a series of papers entitled “On the Constitution of Atoms and Molecules,” Niels Bohr proposes his model of the atom, with electrons making discrete jumps between quantized energy levels.\\
1913&Henry Moseley's paper “The High-Frequency Spectra of the Elements” suggests a physical meaning to atomic number. \\
1914&James Franck and Gustav Hertz's paper “Über Zusammenstöße zwischen Elektronen und Molekülen des Quecksilberdampfes und die Ionisierungsspannung desselben” (“On Collisions Between Electrons and Molecules of Mercury Vapor and the Ionization Potential of the Same”) shows evidence for quantized atomic energy levels.\\
1914&Robert Millikan's paper “A Direct Determination of h” reports his careful replication of the photoelectric effect in a vacuum—an experiment that, much to Millikan's chagrin, validated Einstein's explanation\\
1915&In the paper “Die Feldgleichungen der Gravitation” (“The Field Equations of Gravitation”), Einstein proposes his general theory of relativity, which incorporates gravity into the theory of relativity.\\
1915&Emmy Noether proves that every continuous symmetry of a physical system corresponds to a conservation law. (The proof is not published until three years later.)\\
1920&Dyson, Eddington, and Davidson's paper “A Determination of the Deflection of Light by the Sun's Gravitational Field, From Observations Made at the Total Eclipse of 29 May 1919” provides observational verification of Einstein's general theory of relativity.\\
1922&Otto Stern and Walther Gerlach demonstrate quantization of angular momentum in the paper “Der experimentelle Nachweis der Richtungsquantelung im Magnetfeld” (“The Experimental Proof of Directional Quantization in the Magnetic Field”). It was later recognized that the angular momentum they had measured was spin.\\
1923&Arthur Compton's paper “A Quantum Theory of the Scattering of X-Rays by Light Elements” describes changes in wavelength from light scattering off electrons.\\
1923&Edwin Hubble reports in the paper “Cepheids in Spiral Nebulae” measurements of distances to Cepheid variable stars in nebulae. He proves that some nebulae such as Andromeda are outside our galaxy, thus proving for the first time that anything exists outside our galaxy.\\
1924&Louis de Broglie, in his PhD thesis “Recherches sur la théorie des quanta” (“Research on Quantum Theory”), proposes matter waves.\\
1925&Wolfgang Pauli proposes the exclusion principle in the paper “Über den Einfluss der Geschwindigkeitsabhängigkeit der Elektronenmasse auf den Zeemaneffekt” (“On the Connection Between the Termination of the Electron Groups in the Atom and the Complex Structure of the Spectra”).\\
1925&Goudsmit and Uhlenbeck propose spin (intrinsic angular momentum) to explain the Stern-Gerlach results in “Ersetzung der Hypothese vom unmechanischen Zwang durch eine Forderung bezüglich des inneren Verhaltens jedes einzelnen Elektrons” (“Replacement of the Hypothesis of Nonmechanical Connection by an Internal Degree of Freedom of the Electron”).\\
1926&Erwin Schrödinger's paper “Quantisierung als Eigenwertproblem” (“Quantization as an Eigenvalue Problem”) presents what we now call Schrödinger's equation.\\
1926&Max Born proposes the probabilistic interpretation of wavefunctions in “Zur Quantenmechanik der Stoßvorgänge” (“On the Quantum Mechanics of Collision Processes”).\\
1927&Werner Heisenberg proposes the uncertainty principle in “Über den anschaulichen Inhalt der quantentheoretischen Kinematik und Mechanik” (“About the Descriptive Content of Quantum Theoretical Kinematics and Mechanics”)\\
1927&Davisson and Germer (“Diffraction of Electrons by a Crystal of Nickel”) and G.P. Thomson (“Experiments on the Diffraction of Cathode Rays,” published in 1928) independently demonstrate electron diffraction.\\
1928&Paul Dirac's paper “The Quantum Theory of the Electron” expands the theory of quantum mechanics to be compatible with special relativity.\\
1929&In the paper “A Relation Between Distance and Radial Velocity Among Extra-Galactic Nebulae,” Edwin Hubble shows that most galaxies are receding from us at speeds proportional to their distances from us. This is now generally considered the observational discovery of the expansion of the universe.\\
1931&Georges Lemaître describes his theory of the “Primeval Atom,” now known as the “Big Bang,” to explain Hubble's observations: “A Homogeneous Universe of Constant Mass and Increasing Radius Accounting for the Radial Velocity of Extra-Galactic Nebulae.”\\
1932&James Chadwick's paper “Existence of a Neutron” announces the discovery of the neutron.\\
1933&Carl Anderson's paper “The Positive Electron” announces the discovery of the positron (from work performed at the same lab as Chadwick's discovery of the neutron).\\
1938-1939&Fission of uranium atoms is observed by Otto Hahn and Fritz Strassman, and explained and confirmed by Lise Meitner and Otto Frisch. \\
1949&The first published Feynman diagram appears in the paper “Space-Time Approach to Quantum Electrodynamics.” We can't possibly choose a date when “quantum field theory was invented” but this seems like a nice symbolic milestone.\\
1957&Chien-Shiung Wu's paper “Experimental Test of Parity Conservation in Beta Decay” demonstrates that parity is not an exact symmetry.\\
1961&Claus Jönsson reports a double-slit experiment with electrons in “Elektroneninterferenzen an mehreren künstlich hergestellten Feinspalten” (“Electron Interference at Several Artificially Produced Fine Gaps”).\\
1964&Murray Gell-Mann (“A Schematic Model of Baryons and Mesons”) and George Zweig (“An SU(3) Model for Strong Interaction Symmetry and its Breaking”) independently propose the quark model for hadrons, paving the way for the standard model of particle physics.\\
1964&John Bell's paper “On the Einstein Podolsky Rosen Paradox” demonstrates that quantum mechanics cannot be reconciled with locality.\\
1964&Robert Wilson and Arno Penzias publish “A Measurement Of Excess Antenna Temperature at 4080 Mc/s,” reporting measurements of microwave radiation coming to us from all directions. This is now accepted to be the “cosmic micrwave background” radiation left over from the early universe.\\
1970-1975&Standard Model of Particle Physics. While there is no one moment when this theory was created, the main pieces that brought it into essentially its current form were developed during this time. Those pieces included a better theoretical understanding of strong forces, and the experimental confirmation of the existence of quarks.\\
1980&Vera Rubin and her collaborators publish “Rotational Properties of 21 SC Galaxies With a Large Range of Luminosities and Radii, From NGC
4605 (R=4kpc) to UGC 2885 (R=122kpc),” reporting rotation cureves of 21 galaxies that unambiguously showed the existence of dark matter. This paper was the culmination of work Rubin had published for a number of years before this.\\
1981&Alan Guth publishes “Inflationary Universe: A Possible Solution to the Horizon and Flatness Problems,” introducing the theory of inflation to explain a number of puzzles in Big Bang cosmology.\\
1992&Radio astronomers Aleksander Wolszczan and Dale Frail's paper “A Planetary System Around the Millisecond Pulsar PSR1257 + 12” announces the first definitive detection of planets outside our solar system. (Thirty years later, over 5000 exoplanets have been confirmed.) \\
1998&Adam Riess and collaborators (“Observational Evidence From Supernovae for an Accelerating Universe and a Cosmological Constant”), and soon after Saul Perlmutter and collaborators (“Measurements of Omega
and Lambda from 42 High Redshift Supernovae” in 1999), independently publish supernova measurements showing that the expansion of the universe is accelerating.\\
2012&Detection of the Higgs boson (“Observation of a New Particle in the Search for the Standard Model Higgs Boson with the ATLAS Detector at the LHC”) marks experimental validation of the last piece of the standard model. The paper lists approximately 3000 co-authors.\\
2016&B. P. Abbott et al's paper “Observation of Gravitational Waves from a Binary Black Hole Merger” announces the 2015  detection of gravitational waves, as predicted by Einstein's general theory of relativity, by the Laser Interferometer Gravitational-Wave Observatory (LIGO).\\
2021&In a series of papers, Fermilab announces the results of its “Muon g-2” experiment, showing that the measured gyromagnetic ratio (or “g-factor”) of the muon differs from theoretical prediction with a significance of 4.2 sigma. If confirmed, this measurement will be the first-ever experimental failure of the standard model of particle physics.
    \end{longtable}
\end{center}

\end{document}