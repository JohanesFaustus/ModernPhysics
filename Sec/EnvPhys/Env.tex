\documentclass[../main.tex]{subfiles}

\begin{document}
\subsection*{Intro}
Homeotherm: kemampuan tubuh untuk mempertahakan suhu tubuhu di dalam perubahan kondisi lingkungan

Bumi menyediakan bantuan kepada makhluk hidup untuk tumbuh dan berkembang

Hukum I thermo dalam konteks tubuh
\begin{equation*}
    dM=dH+dW
\end{equation*}
denan $dW$ sebagai metabolic rate, $dH$ sebagai total reproduksi energi dari metabolisme tubuh, dan $dW$ sebagai kerja eksternal oleh tubuh. 

\subsection{Basal Metabolic Rate}
Yaitu laju metabolisme pada kondisi puasa dan mengasilkan enegi yang tidak cukup. Terdapat kerja dari tubuh unutk mejaga fungis vital organ. Setara dengan laju metabolisme saar tidur. Pada saat ini tubuh berada pada kondisi optimal untuk memperbaiki tubuh karena metabolisme tidak diguakan untuk mencerna makanan.

Berikut disipasi enegi
\begin{enumerate}
    \item Sleeping: 75 W 
    \item Sitting: 80-100 W 
    \item Walking: 150-450 W 
    \item Running hard: 400-1500 W 
\end{enumerate}

\subsection*{Transfer enegi}
Terdapat 4 metode: konduksi, konveksi, evaporasi, dan radiasi. Hukum kekekalan energi 
\begin{align*}
    E_\text{in}=E_\text{out}\\
    M+R_\text{ext}=R+C_\text{kond}+C_\text{konv}
\end{align*}

\subsubsection*{Konduksi.} Umumnya terjadi pada padatan. Rumusan Fourier: aliran thermal bergantung pada luas penampang $A$, panjang bahan $L$, dan perbedaan suhu keduasisi $\Delta T$. 
\begin{equation*}
    \frac{dQ}{dt}=-\frac{kA}{L}dT;\quad\text{dimana }dT=T_1-T_2,T_1>T_2
\end{equation*}
$k$ merupakan konstanta konduktivitas bahan. 


\subsubsection*{Konveksi.}
Pada umumnya terjadi pada fluida. Hukum pendinginan Newton 
\begin{equation*}
    \frac{dQ}{dt}=-kA(T-T_C)
\end{equation*}
dimana $T_C$ sebagai suhu lingkungan.
\subsubsection*{Evaporasi.} Metode mempertahankan suhu tubuh. Mode evaporasi 
\begin{equation*}
    \frac{dQ}{dt}=hA(P_s-P_0)
\end{equation*}
dimana h adalah koefisien energi evaporasi, $P_s$ tekanan air keringat, dan $P_0$ tekanan udara sekitar.

\subsubsection*{Radiasi.} Perpindahan energi via gelombang elektromagnetik. Tubuh memancarkan GEM dengan $\lambda$ setara dengan infrared

\subsection*{Perbandingan tubuh manusia dan hewan}
\subsubsection*{Manusia.} Boundary layer is skin.
\subsubsection*{Animal.} Boundary layer is fur.

\subsection*{Stefan Boltzman Law}
Besar enegi
\begin{eqnarray}
    E=\epsilon\sigma T^4
\end{eqnarray}
Rasio dari radiasi diserap adalah
\begin{equation*}
    E-E_{\text{absorbed}}=\epsilon \sigma (T^4-T_0^4)
\end{equation*}
dan besarnya daya radiasi 
\begin{equation*}
    P=\epsilon A \sigma T^4
\end{equation*}

\section*{Case Base} Buat makalah dengan topik
\begin{enumerate}
\item Bagaimana mekanisme tubuh manusia mengarut suhu
\item Bagaimana manusia memepertahankan kenyamanan dalam lingkungan ekstrim: kenyamanan thermal dan insulasi panas
\end{enumerate}
\end{document}