\documentclass[../main.tex]{subfiles}
\begin{document}
\subsection*{Appendix: Tugas 1 Gelombang 29 Agustus 2024}

\begin{figure}[b]
    \centering
    \includesvg[height=0.4\textwidth, inkscape=force]{../Rss/Waves/Appen/GHSPend.svg}
    \includesvg[width=0.4\textwidth, inkscape=force]{../Rss/Waves/Appen/GHSTali.svg}
    \caption*{Gambar: GHS pada ayunan massa (soal 1) dan massa pada tali (soal 2).}
  \end{figure}
  \subsubsection*{Soal 1: Massa dalam pendulum.} Mengingat bahwa $  \boldsymbol{\tau} =I\boldsymbol{\ddot{\theta}} = \mathbf{r}\times \mathbf{F}$, maka resultan torsi yang bekerja pada sistem adalah
  \begin{align*}
    \sum I\boldsymbol{\ddot{\theta}}&=\sum( \mathbf{r}\times \mathbf{F})\\
    I_b\boldsymbol{\ddot{\theta}}+I_p\boldsymbol{\ddot{\theta}}&=\mathbf{r}_b\times \mathbf{F}_b+\mathbf{r}_p\times \mathbf{F}_p\\
    (I_b+I_p)\ddot{\theta}&=(\frac{l}{2})m_bg\sin(\pi+\theta)+(l+R)m_pg\sin(\pi+\theta)\\
    \ddot{\theta}&=\frac{1}{I_b+I_p}(\frac{l}{2}m_bg+ (l+R)m_pg) (-\sin\theta)\\
    \ddot{\theta}&=-\frac{lm_bg+2lm_pg+2Rm_pg}{2(I_b+I_p)}\sin\theta
  \end{align*}
  Menggunakan perkiraan sudut kecil $\sin x\approx x$,
  \begin{equation*}
    \ddot{\theta}=-\frac{lm_bg+2lm_pg+2Rm_pg}{2(I_b+I_p)}\theta
  \end{equation*}
  Diketahui bahwa momen inersia batang $I_b=\frac{1}{3}m_bl^2$ dan piringan $I_p=\frac{1}{2}mR^2$, maka
  \begin{align*}
    \ddot{\theta}&=-\frac{lm_bg+2lm_pg+2Rm_pg}{2(\frac{1}{3}m_bl^2+\frac{1}{2}m_pR^2)}\theta\\
    &=-\frac{2gl(m_b/l+m_p+m_pR/l)}{(2m_bl^2+3m_pR^2)/6}\theta\\
    &=-\frac{6gl}{l^2}\frac{m_b/l+m_p(1+R/l)}{(2m_b+3m_pR^2/l^2)}\theta\\
    \ddot{\theta}&=-\frac{6g}{l}\frac{m_b/l+m_p(1+R/l)}{(2m_b+3m_pR^2/l^2)}\theta
  \end{align*}
  Persamaan tersebut adalah persamaan GHS $\ddot{\theta}=-\omega^2\theta$ dengan
  \begin{equation*}
    \omega=\sqrt{\frac{6g}{l}\frac{m_b/l+m_p(1+R/l)}{(2m_b+3m_pR^2/l^2)}}
  \end{equation*}
  
  Pembuat jam pendulum akan membuat jam sedemikian rupa sehingga periode pendulum adalah 1 detik. Karena $T=2\pi/\omega$, $\omega$ jam tersebut adalah $2\pi$. Dengan demikian, persamaan
  \begin{equation*}
    2\pi=\sqrt{\frac{6g}{l}\frac{m_b/l+m_p(1+R/l)}{(2m_b+3m_pR^2/l^2)}}
  \end{equation*}
  harus dipenuhi sehingga jam pendulum memiliki periode sebesar 1 detik.
  
  \subsubsection*{Soal 2: Massa pada tali.} Menggunakan hukum Newton pada sistem, diperoleh
  \begin{align*}
    \sum \mathbf{F}_y&=m\boldsymbol{\ddot{y}}\\
    m\boldsymbol{\ddot{y}}&=-\mathbf{T}_1\sin\theta_1-\mathbf{T}_2\sin\theta_2\\
  \end{align*}
  Menggunakan perkiraan sudut kecil $\sin x\approx\tan x$,
  \begin{align*}
    m\ddot{y}&=-{T}_1\frac{y}{a}-{T}_2\frac{y}{2l-a}\\
    \ddot{y}&=-\biggl(\frac{{T}_1}{ma}+\frac{{T}_2}{m(2l-a)}\biggr)y
  \end{align*}
  Persamaan tersebut adalah persamaan GHS $\ddot{\theta}=-\omega^2\theta$ dengan 
  \begin{equation*}
    \omega=\sqrt{\frac{{T}_1}{ma}+\frac{{T}_2}{m(2l-a)}}
  \end{equation*}
  Sebagai cek, diketahui ketika $a=l$, $\omega=\sqrt{2T/(ml)}$. Ketika $a=l$, $T_1=T_2$; maka berdasarkan persamaan,
  \begin{align*}
    \omega_{a\rightarrow l}&=\sqrt{\frac{{T}}{ml}+\frac{{T}}{m(2l-l)}}\\
    \omega_{a\rightarrow l}&=\sqrt{\frac{2T}{ml}}
  \end{align*}
  sesuai dengan nilai $\omega$ ketika $a=l$


\end{document}